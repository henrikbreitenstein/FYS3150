\documentclass[english,notitlepage]{revtex4-1}  % defines the basic parameters of the document
%For preview: skriv i terminal: latexmk -pdf -pvc filnavn



% if you want a single-column, remove reprint

% allows special characters (including æøå)
\usepackage[utf8]{inputenc}
%\usepackage[english]{babel}

%% note that you may need to download some of these packages manually, it depends on your setup.
%% I recommend downloading TeXMaker, because it includes a large library of the most common packages.

\usepackage{physics,amssymb}  % mathematical symbols (physics imports amsmath)
\include{amsmath}
\usepackage{graphicx}         % include graphics such as plots
\usepackage{xcolor}           % set colors
\usepackage{hyperref}         % automagic cross-referencing (this is GODLIKE)
\usepackage{listings}         % display code
\usepackage{subfigure}        % imports a lot of cool and useful figure commands
\usepackage{float}
%\usepackage[section]{placeins}
\usepackage{algorithm}
\usepackage[noend]{algpseudocode}
\usepackage{subfigure}
\usepackage{tikz}
\usetikzlibrary{quantikz}
% defines the color of hyperref objects
% Blending two colors:  blue!80!black  =  80% blue and 20% black
\hypersetup{ % this is just my personal choice, feel free to change things
    colorlinks,
    linkcolor={red!50!black},
    citecolor={blue!50!black},
    urlcolor={blue!80!black}}

%% Defines the style of the programming listing
%% This is actually my personal template, go ahead and change stuff if you want



%% USEFUL LINKS:
%%
%%   UiO LaTeX guides:        https://www.mn.uio.no/ifi/tjenester/it/hjelp/latex/
%%   mathematics:             https://en.wikibooks.org/wiki/LaTeX/Mathematics

%%   PHYSICS !                https://mirror.hmc.edu/ctan/macros/latex/contrib/physics/physics.pdf

%%   the basics of Tikz:       https://en.wikibooks.org/wiki/LaTeX/PGF/Tikz
%%   all the colors!:          https://en.wikibooks.org/wiki/LaTeX/Colors
%%   how to draw tables:       https://en.wikibooks.org/wiki/LaTeX/Tables
%%   code listing styles:      https://en.wikibooks.org/wiki/LaTeX/Source_Code_Listings
%%   \includegraphics          https://en.wikibooks.org/wiki/LaTeX/Importing_Graphics
%%   learn more about figures  https://en.wikibooks.org/wiki/LaTeX/Floats,_Figures_and_Captions
%%   automagic bibliography:   https://en.wikibooks.org/wiki/LaTeX/Bibliography_Management  (this one is kinda difficult the first time)
%%   REVTeX Guide:             http://www.physics.csbsju.edu/370/papers/Journal_Style_Manuals/auguide4-1.pdf
%%
%%   (this document is of class "revtex4-1", the REVTeX Guide explains how the class works)


%% CREATING THE .pdf FILE USING LINUX IN THE TERMINAL
%%
%% [terminal]$ pdflatex template.tex
%%
%% Run the command twice, always.
%% If you want to use \footnote, you need to run these commands (IN THIS SPECIFIC ORDER)
%%
%% [terminal]$ pdflatex template.tex
%% [terminal]$ bibtex template
%% [terminal]$ pdflatex template.tex
%% [terminal]$ pdflatex template.tex
%%
%% Don't ask me why, I don't know.

\begin{document}

\title{Title of the document}      % self-explanatory
\author{Your name(s) here}          % self-explanatory
\date{\today}                             % self-explanatory
\noaffiliation                            % ignore this, but keep it.


\maketitle 
    
\textit{List a link to your github repository here!}
    
\section*{Problem 1}

\begin{center}
	\textit{https://github.com/henrikbreitenstein/FYS3150.git}
\end{center}
\section*{Problem 1}

Poisson equation
$$- \frac{\mathrm{d}^2 u}{\mathrm{d}x^2} = f(x)$$

Replaceing $f(x)$ with given function

\begin{align*}
- \frac{\mathrm{d}^2 u}{\mathrm{d}x^2} =& 100e^{-10x} \\
-\mathrm{d}^2 u =& 100e^{-10x} \; \mathrm{d}x^2
\end{align*}
Taking integrals

\begin{align*}
-\int \int \mathrm{d}^2 u =& \int \int 100e^{-10x} \; \mathrm{d}x^2 \\
-u =& \int -10e^{-10x} + c_1 \; \mathrm{d} x \\
-u =& e^{-10x} + c_1x + c_2 \\
u =& -e^{-10x} - c_1x - c_2 \\
\end{align*}
Using initial conditions:


\begin{equation}\label{eq1}
u(0) = 0 \Rightarrow - 1 - c_2 = 0
\end{equation}


\begin{equation}\label{eq2}
u(1) = 0 \Rightarrow -e^{-10} -c1 -c2 = 0
\end{equation}

With \ref{eq1} and \ref{eq2} we get:

\begin{align*}
c2 =& -1 \\
c1 =& 1 - e^{-10}
\end{align*}
By replacing $c_1$ and $c_2$ we get:

\begin{equation}\label{equ}
u = 1 - (1-e^{-10})x - e^{-10x} 
\end{equation}


\section*{Problem 2}
We write equations using the LaTeX \texttt{equation} (or \texttt{align}) environments. Here is an equation with numbering
\begin{equation}\label{eq:newton}
    \vb{F} = \dv{\vb{p}}{t},
\end{equation}
and here is one without numbering:

\section*{Problem 5}
\subsection{Problem b}
Av $A\vec{v}=\vec{g}$ vil vi finne alle verdier mellom grensebetingelsene $u(0)=u(1)=0$. $\vec{v*}$ er den samme som $\vec{v}$ men vi har lagt til $v_0$ og $v_{n+1}$ som er grensebetingelsene.

\section{Problem 6}
\section*{a}
Vi har nå en vektor
$$
\vec{v}=\begin{pmatrix}
v_1 \\ v_2 \\ \vdots \\ v_n
\end{pmatrix}
$$
og en $n\cross n$-matrise
$$
\textbf{A}=\begin{Bmatrix}
b_1 & c_2 & 0 & 0 & \cdots & 0 &0 \\
a_2 & b_2 & c_2 & 0 & \cdots &0 &0 \\
0 & a_3 & b_3 & c_3 & \cdots & 0 &0 \\
\vdots & \vdots&\vdots&\vdots&\ddots & \vdots & \vdots\\
0& \cdots &  \cdots&\cdots&\cdots& a_n & b_n
\end{Bmatrix}
$$
Matrisemultipliserer vi disse får vi
$$
\textbf{A}\vec{v}=
\begin{matrix}
(I)&b_1 v_1 &+ c_1 v_2 & &&=g_1 \\
(II) &a_2 v_1&+b_2v_2&+c_3 v_3 &&=g_2\\
(III)& & a_3v_2 &+b_3v_3 &+ c_3 v_4 &=g_3\\
\vdots&&&&&\\
(n)&&&a_n v_{n-1}&+b_n v_n&=g_n\\
 
\end{matrix}
$$ Herfra skal vi radredusere og starter med å ta $(II)-\frac{a_2}{b_1}(I)$ slik at vi får $$\begin{matrix}
(II*) & 0&b_2-\frac{c_1\cdot a_2}{b_1}& c_2 &\cdots & g_2-g_1\frac{a_2}{b_1}
\end{matrix}$$
og vi ser da at $a_2$ går bort. Vi kan også sette $b_2^*\equiv b_2-\frac{a_2 \cdot c_1}{b_1}$ og $g_2^*\equiv g_2-g_1\frac{a_2}{b_1}$. Da har vi mellom $(II^*)$ og $(III)$ noe som ser ganske likt ut som det vi hadde mellom $(I)$ og $(II)$. Derfor gjør vi det samme som vi gjorde før og tar $(III)-\frac{a_3}{b_2^*}(II)^*$ og får 
$$
\begin{matrix}
(III)^* & 0 &0& b_3-\frac{c_2\cdot a_3}{b_2^*}&c_3 & g_3-g_1\frac{a_3}{b_2^*}
\end{matrix}
$$
og vi kan igjen definere $b_3^*=b_3-\frac{c_2\cdot a_3}{b_2^*}$ og $g_3^*=g_3-g_1\frac{a_3}{b_2^*}$. Og vi kan da fortsette med dette nedover som $(k)-\frac{a_{k}}{b_{k-1}^*}(k-1)^*$. Så har vi fjernet $a$-ene så da må vi fjerne $c$-ene. Vi starter nå på siste og neste siste rad, altså $(n)^*$ og $(n-1)^*$ som nå er
$$
\begin{matrix}
(n-1)^* & 0& \cdots & b_{n-1}^* &c_{n-1}&g_{n-1}^* \\
(n)^* & 0 & \cdots & 0 & b_{n}^*& g_{n-1}^*
\end{matrix}
$$
så hvis vi da tar $(n-1)^*-\frac{c_{n-1}}{b_n^*}(n)^*$ får vi
$$
\begin{matrix}
\tilde{(n-1)} & 0 & \cdots & b_{n-1}^* & 0 & g_{n-1}^*-\frac{c_{n-1} g_n^*}{b_n^*}
\end{matrix}
$$
og vi setter $\tilde{g_{n-1}}= g_{n-1}^*-\frac{c_{n-1} g_n^*}{b_n^*}$ og dette gjør vi videre oppover som $(k-1)-\frac{c_k}{b_{k-1}}(k)$
Til slutt står vi da bare igjen med $b^*$-ene og disse kan vi da dele på seg selv og vi får en identitetsmatrise.
Vi kan nå skrive dette som en algoritme. Anta vi har en tridiagonal matrise
$$
A=U=\begin{pmatrix}
u_{1,1}&u_{1,2}&\cdots & u_{1,n} \\
u_{2,1}&u_{2,2}&\cdots & u_{2,n} \\
\vdots & \vdots& \ddots & \vdots \\
u_{n,1} & u{n,2}& \cdots & u{n,n}
\end{pmatrix}
$$
og en som skal løses for vektoren
$$
g=h=\begin{pmatrix}
h_1\\h_2\\\vdots \\ h_n
\end{pmatrix}
$$

\begin{algorithm}[H]
	\caption{Radredusering av tridiagonal matrise}\label{algo:midpoint_rule}
	\begin{algorithmic}
		\For{$i = 2, 3, ..., n$} \Comment{Forward substitution, $n-1$ repetisjoner}
		\State $$t\leftarrow \frac{u_{i,i-1}}{u{_i-1,i}}$$ \Comment{1 FLOP}
		$$
		u_{i,i-1}\leftarrow u_{i,i-1}-u_{i-1,i-1} \cdot t
		$$ \Comment{2 FLOPs}
		$$
		u_{i,i}\leftarrow u_{i,i}-u_{i-1,i-1}\cdot t
		$$ \Comment{2 FLOPs}
		$$
		h_i\leftarrow h_i-h_{i-1} t
		$$ \Comment{2 FLOPs} \\
		\EndFor
		\Comment{Til sammen $7\cdot (n-1)$ FLOPs i loopen}
		\For{$j=n-1,n-2, ... ,1$} \Comment{Backward Substitution, $n-1$ repetisjoner}
		\State $$k\leftarrow \frac{u_{j,j+1}}{u_{j+1,j}}$$ \Comment{2 FLOPs}
		$$
		u_{j,j+1}\leftarrow u_{j,j+1}-u_{j+1,j+1}\cdot k
		$$ \Comment{2 FLOPs}
		$$
		u_{j,j}\leftarrow u_{j,j}-u_{j+1,j}\cdot k
		$$ \Comment{2 FLOPs}
		$$
		h_j\leftarrow h_j-h_{j+1}\cdot k
		$$ \Comment{2 FLOPs} \\
		\EndFor
		\Comment{Til sammen $7\cdot (n-1)$ FLOPs i loopen}
		\For {$l=1,2,\cdots n$} \Comment{Deler for å få identitetsmatrise, til sammen $n$ repetisjoner}
		\State $$
		u_l \leftarrow \frac{u_l}{u_l}$$ \Comment{1 FLOP}
		$$
		h_l\leftarrow \frac{h_l}{u_l}
		$$ \Comment{1 FLOPs} \\
		\EndFor
		\Comment{Til sammen $2\cdot n$ FLOPs}
	\end{algorithmic}
\end{algorithm}
Da ser vi at vi til sammen får $2\cdot n+ 2\cdot 7\cdot (n-1)=16n-14$ FLOPs.









Sometimes it is useful to refer back to a previous equation, like we're demonstrating here for equation \ref{eq:newton}.

We can include figures using the \texttt{figure} environment. Whenever we include a figure or table, we \textit{must} make sure to actually refer to it in the main text, e.g.\ something like this: ``In figure \ref{fig:rel_err} we show \ldots''. 

Also, note the LaTeX code we used to get correct quotation marks in the previous sentence. (Simply using the \texttt{"} key on your keyboard will give the wrong result.) Figures should preferably be vector graphics (e.g.\ a \texttt{.pdf} file) rather than raster graphics (e.g.\ a \texttt{.png} file).

By the way, don't worry too much about where LaTeX decides to place your figures and tables --- LaTeX knows more than we do about proper document layout. As long as you label all your figures and tables and refer to them in the text, it's all good. Of course, in some cases it can be worth trying to force a specific placement, to avoid the figure/table appearing many pages away from the main text discussing it, but this isn't something you should spend time on until the very end of the writing process.


Next up is a table, created using the \texttt{table} and \texttt{tabular} environments. We refer to it by table \ref{tab:output_table}.
\begin{table}%[h!]
    \centering
    \begin{tabular}{c@{\hspace{1cm}} c}
        \hline
        Number of points & Output \\
        \hline
        10 &  0.3086\\
        100 &  0.2550\\
        \hline
    \end{tabular}\caption{Write a descriptive caption here, explaining the content of your table.}\label{tab:output_table}
\end{table}

Finally, we can list algorithms by using the \texttt{algorithm} environment, as demonstrated here for algorithm \ref{algo:midpoint_rule}.

\end{document}
